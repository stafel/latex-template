\documentclass[11pt, a4paper]{article}

\usepackage[a4paper, left=25mm, top=30mm, right=25mm, bottom=30mm]{geometry}

\usepackage{times} % set font typeface to Times New Roman

\usepackage{etoolbox}
\usepackage[ngerman]{babel} % make it german
\usepackage{xspace}
\usepackage{graphicx}
\usepackage{blindtext} % for lorem impsum
\usepackage{multicol} % allow arbitrary number of columns
\usepackage[dvipsnames, table]{xcolor}
\usepackage{fancyhdr}
\usepackage{lastpage}
\usepackage{eso-pic,graphicx}
\usepackage{float} % for H setting to include float in multicol
\usepackage{paralist} % compact list
\usepackage{framed}
\usepackage{csvsimple}
\usepackage{listofitems}
\usepackage{etoolbox}
\usepackage{microtype} % helps with justification
\usepackage{hyperref}
\usepackage{vhistory} % version history
\usepackage{enumitem} % allows nolistsep, noitemsep to reduce spacing in listings
\usepackage{listings} % used to hightlight code examples
\usepackage{tabularx}
\usepackage{enumitem}  % for enumerate environmet which can define start and label
\usepackage{tikz}
\usetikzlibrary{positioning}
\usepackage[backend=bibtex,style=alphabetic]{biblatex} % verbose-trad2 alphabetic verbose
\usepackage[toc]{glossaries}

\hypersetup{
	hidelinks=true,
	colorlinks,
	allcolors=Blue
}

% define duo minipage for allignment
\newcommand{\duoalign}[4]{
	\begin{minipage}[t]{#1\textwidth}
		\textbf{#2}
	\end{minipage}
	\begin{minipage}[t]{#3\textwidth}
		#4\\
	\end{minipage}
}

% command to define the author names to reference in the document
\newcommand{\authornames}[1]{\author{#1} \renewcommand{\gauthornames}{#1}}
\newcommand{\gauthornames}{SET AUTHOR!}

% command to define the pdf name to reference in the document
\newcommand{\docname}[1]{\renewcommand{\gdocname}{#1}}
\newcommand{\gdocname}{DOCNAME REQUIRED!}

% command to define the logo in the header
\newcommand{\headerlogo}[1]{\renewcommand{\gheaderlogo}{#1}}
\newcommand{\gheaderlogo}{res/headerlogo}

% Set the page style to "fancy"...
\pagestyle{fancy}
\fancyhf{}
\lhead{\includegraphics[width=.3\linewidth]{\gheaderlogo}}
\rhead{\nouppercase{\leftmark}} % use \nouppercase{\rightmark} for subsection
%\rfoot{Page \thepage \hspace{1pt} of \pageref{LastPage}}
\rfoot{Seite \thepage \xspace von \pageref{LastPage}}
\lfoot{\gdocname}

\fancypagestyle{plain}{ % this is used to redefine 'plain' pages like table of contents
	\fancyhf{}
	\lhead{\includegraphics[width=.3\linewidth]{\gheaderlogo}}
	\rhead{\nouppercase{\leftmark}} % use \nouppercase{\rightmark} for subsection
	%\rfoot{Page \thepage \hspace{1pt} of \pageref{LastPage}}
	\rfoot{Seite \thepage \xspace von \pageref{LastPage}}
	\lfoot{\gdocname}
}

% command to define event date used in the protocol
\newcommand{\eventdate}[1]{\renewcommand{\geventdate}{#1}}
\newcommand{\geventdate}{\today}

% command to define event timeframe used in the protocol
\newcommand{\eventtime}[1]{\renewcommand{\geventtime}{#1}}
\newcommand{\geventtime}{SET EVENT TIME!}

% command to define event leader used in the protocol
\newcommand{\eventleader}[1]{\renewcommand{\geventleader}{#1}}
\newcommand{\geventleader}{SET EVENT LEADER!}

% command to define event place used in the protocol
\newcommand{\eventplace}[1]{\renewcommand{\geventplace}{#1}}
\newcommand{\geventplace}{SET EVENT PLACE!}

% command to define event distribution list used in the protocol, all participants and absentees are added automaticaly
\newcommand{\geventdistributions}{}
\newcommand{\eventdistribution}[1]{%
	\ifdefempty{\geventdistributions}
	{\appto\geventdistributions{#1}}
	{\appto\geventdistributions{\\#1}}%
}

% command to define event participants list used in the protocol
\newcommand{\geventparticipants}{}
\newcommand{\eventparticipant}[1]{%
	\eventdistribution{#1} % add to distribution list
	\ifdefempty{\geventparticipants}
	{\appto\geventparticipants{#1}}
	{\appto\geventparticipants{\\#1}}%
}

% command to define event absentees list used in the protocol
\newcommand{\geventabsentees}{}
\newcommand{\eventabsentee}[1]{%
	\eventdistribution{#1} % add to distribution list
	\ifdefempty{\geventabsentees}
	{\appto\geventabsentees{#1}}
	{\appto\geventabsentees{\\#1}}%
}

% command to define event agenda items used in the protocol
\newcommand{\geventagenda}{}
\newcommand{\eventagenda}[1]{%
	\ifdefempty{\geventagenda}
	{\appto\geventagenda{#1}}
	{\appto\geventagenda{\\#1}}%
}