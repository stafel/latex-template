


\documentclass[11pt, a4paper]{article}

\usepackage[a4paper, left=25mm, top=30mm, right=25mm, bottom=30mm]{geometry}

\usepackage{times} % set font typeface to Times New Roman

\usepackage{etoolbox}
\usepackage[ngerman]{babel} % make it german
\usepackage{xspace}
\usepackage{graphicx}
\usepackage{blindtext} % for lorem impsum
\usepackage{multicol} % allow arbitrary number of columns
\usepackage[dvipsnames, table]{xcolor}
\usepackage{fancyhdr}
\usepackage{lastpage}
\usepackage{eso-pic,graphicx}
\usepackage{float} % for H setting to include float in multicol
\usepackage{paralist} % compact list
\usepackage{framed}
\usepackage{csvsimple}
\usepackage{listofitems}
\usepackage{etoolbox}
\usepackage{microtype} % helps with justification
\usepackage{hyperref}
\usepackage{vhistory} % version history
\usepackage{enumitem} % allows nolistsep, noitemsep to reduce spacing in listings
\usepackage{listings} % used to hightlight code examples
\usepackage{tabularx}
\usepackage{enumitem}  % for enumerate environmet which can define start and label
\usepackage{tikz}
\usetikzlibrary{positioning}
\usepackage[backend=bibtex,style=alphabetic]{biblatex} % verbose-trad2 alphabetic verbose
\usepackage[toc]{glossaries}

\hypersetup{
	hidelinks=true,
	colorlinks,
	allcolors=Blue
}

% define duo minipage for allignment
\newcommand{\duoalign}[4]{
	\begin{minipage}[t]{#1\textwidth}
		\textbf{#2}
	\end{minipage}
	\begin{minipage}[t]{#3\textwidth}
		#4\\
	\end{minipage}
}

% command to define the author names to reference in the document
\newcommand{\authornames}[1]{\author{#1} \renewcommand{\gauthornames}{#1}}
\newcommand{\gauthornames}{SET AUTHOR!}

% command to define the pdf name to reference in the document
\newcommand{\docname}[1]{\renewcommand{\gdocname}{#1}}
\newcommand{\gdocname}{DOCNAME REQUIRED!}

% command to define the logo in the header
\newcommand{\headerlogo}[1]{\renewcommand{\gheaderlogo}{#1}}
\newcommand{\gheaderlogo}{res/headerlogo}

% Set the page style to "fancy"...
\pagestyle{fancy}
\fancyhf{}
\lhead{\includegraphics[width=.3\linewidth]{\gheaderlogo}}
\rhead{\nouppercase{\leftmark}} % use \nouppercase{\rightmark} for subsection
%\rfoot{Page \thepage \hspace{1pt} of \pageref{LastPage}}
\rfoot{Seite \thepage \xspace von \pageref{LastPage}}
\lfoot{\gdocname}

\fancypagestyle{plain}{ % this is used to redefine 'plain' pages like table of contents
	\fancyhf{}
	\lhead{\includegraphics[width=.3\linewidth]{\gheaderlogo}}
	\rhead{\nouppercase{\leftmark}} % use \nouppercase{\rightmark} for subsection
	%\rfoot{Page \thepage \hspace{1pt} of \pageref{LastPage}}
	\rfoot{Seite \thepage \xspace von \pageref{LastPage}}
	\lfoot{\gdocname}
}

% command to define event date used in the protocol
\newcommand{\eventdate}[1]{\renewcommand{\geventdate}{#1}}
\newcommand{\geventdate}{\today}

% command to define event timeframe used in the protocol
\newcommand{\eventtime}[1]{\renewcommand{\geventtime}{#1}}
\newcommand{\geventtime}{SET EVENT TIME!}

% command to define event leader used in the protocol
\newcommand{\eventleader}[1]{\renewcommand{\geventleader}{#1}}
\newcommand{\geventleader}{SET EVENT LEADER!}

% command to define event place used in the protocol
\newcommand{\eventplace}[1]{\renewcommand{\geventplace}{#1}}
\newcommand{\geventplace}{SET EVENT PLACE!}

% command to define event distribution list used in the protocol, all participants and absentees are added automaticaly
\newcommand{\geventdistributions}{}
\newcommand{\eventdistribution}[1]{%
	\ifdefempty{\geventdistributions}
	{\appto\geventdistributions{#1}}
	{\appto\geventdistributions{\\#1}}%
}

% command to define event participants list used in the protocol
\newcommand{\geventparticipants}{}
\newcommand{\eventparticipant}[1]{%
	\eventdistribution{#1} % add to distribution list
	\ifdefempty{\geventparticipants}
	{\appto\geventparticipants{#1}}
	{\appto\geventparticipants{\\#1}}%
}

% command to define event absentees list used in the protocol
\newcommand{\geventabsentees}{}
\newcommand{\eventabsentee}[1]{%
	\eventdistribution{#1} % add to distribution list
	\ifdefempty{\geventabsentees}
	{\appto\geventabsentees{#1}}
	{\appto\geventabsentees{\\#1}}%
}

% command to define event agenda items used in the protocol
\newcommand{\geventagenda}{}
\newcommand{\eventagenda}[1]{%
	\ifdefempty{\geventagenda}
	{\appto\geventagenda{#1}}
	{\appto\geventagenda{\\#1}}%
}

\newglossaryentry{test}
{
	name=test,
	description={A check if everything works as expected}
}
\makeglossaries

% set name to show in footer
\docname{documentation.pdf}

% this is the bibligoraphy 
\bibliography{literatur}

% this is the title side definition
\title{\includegraphics[width=\textwidth]{res/logo.png}\\
	Pflichtenheft\\
	\small{Version \vhCurrentVersion}}
\author{SET AUTHOR!}
\date{\vhCurrentDate}

\begin{document}
	\maketitle
	\begin{versionhistory}
	\vhEntry{0.1}{30.03.2024}{RS}{Dokumentstruktur}
\end{versionhistory}
	\tableofcontents
	\newpage
	
	\section{test}
	The quick brown fox jumps over the lazy dog \newpage
	Die von Maas vorgeschlagene Standartisierungsmodell \autocite[12]{ARTICLE:1}  wurde in die Managementbereiche nach dem Funktionsmodell \autocite[35-60]{BOOK:1} unterteilt.  Dabei wurde die Funktionskopfzuteilung beachtet \footcite{WEBSITE:1}.\newpage
	p3 \gls{test} \newpage
	
	\section{Example}

For links see \hyperlink{target}{the aptly named chapter at the end}.

\subsection{Citations}
\label{subsec:citations}

\subparagraph{Citations} are very important like for a book\footcite{SJGTHS:1} or for a website\footcite{COADE:1}. You find them in the \emph{citations.bib} file.

\subsection{Figures}

\begin{figure}[H]
	\centering
	\includegraphics[width=\textwidth]{res/example.png}
	\caption[Short caption]{One figure centered full width}
	\label{fig:bigImage}
\end{figure}

\begin{figure}[H]
	\centering
	\begin{minipage}[b]{0.45\textwidth}
		\includegraphics[width=\textwidth]{res/example.png}
		\caption{Left figure}
	\end{minipage}
	\hfill
	\begin{minipage}[b]{0.45\textwidth}
		\includegraphics[width=\textwidth]{res/example.png}
		\caption{Right figure}
	\end{minipage}
\end{figure}

Notice how we can setup a \emph{label} and then reference the figure \ref{fig:bigImage} on page \pageref{fig:bigImage} with the \emph{ref} command.

\subsection{Tables}

\begin{table}[H]
	\centering
	\resizebox{\textwidth}{!}{ % this neat trick will resize the table down
	\begin{tabular}{ || m{3cm} | m{5cm} | m{8cm} || }
		\hline
		\textbf{A title} & \textbf{B title} & \textbf{C title} \\
		\hline
		\hline
		A & B & C\\
		\hline
		A & B & C\\
		\hline
		A & B & C\\
		\hline
	\end{tabular}
	}
	\caption{Simple table with lines}
	\label{tab:exampleTable}
\end{table}

\subsection{Lists, Bulletpoints and Enumerations}

This itemize list is default.
\begin{itemize}
	\item A
	\item B
	\item C
\end{itemize}

This enumerate list has \emph{noitemsep} added.
\begin{enumerate}[noitemsep]
	\item A
	\item B
	\item C
\end{enumerate}

\hypertarget{target}{\subsection{Links and References}}

A label works great to reference something inside the document like section~\ref{subsec:citations} the citations on page~\pageref{subsec:citations}.\\
For more freedom to navigate consider a hypertarget and hyperlink like this chapter title.
For external webpages use \url{https://en.wikibooks.org/wiki/LaTeX/Labels_and_Cross-referencing} but a citation would be better suited most of the time.

\subsection{Glossary}
This is a \gls{test} if the glossaries package works as expected.
Remember that in TeXstudio you have to generate the glossary with F9 (makeglossaries command) first before it will show up.

	% must be input instead of include else rewriting of aux files is prevented by security settings
	\printglossaries % Glossar, remember to regenerate with F9 in TeXstudio
\printbibliography[heading=bibintoc,title={Quellenverzeichnis}]
\listoffigures
\addcontentsline{toc}{section}{Abbildungsverzeichnis} % fix toc entry Abbildungsverzeichnis
\listoftables
\addcontentsline{toc}{section}{Tabellenverzeichnis} % fix toc entry Tabellenverzeichnis % remember to regenerate glossary with F9 in TeXstudio
\end{document}